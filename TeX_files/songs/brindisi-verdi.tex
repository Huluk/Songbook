\songtitle{Brindisi (Libiamo ne' lieti calici)}{Guiseppe Verdi}{1853}
\medskip

\begin{guitar}
	\songsection{Alfredo}
	Libiamo, libiamo ne’ lieti calici
	che la bellezza infiora.
	E la fuggevol ora s’inebrii a voluttà
	Libiam ne’ dolci fremiti
	che suscita l’amore,
	poiché quell’ochio al core onnipotente va.
	Libiamo, amore, amor fra i calici
	più caldi baci avrà

	\songsection{Coro}
	Ah! Libiam, amor, fra’ calici più caldi baci avrà

	\songsection{Violetta}
	Tra voi tra voi saprò dividere
	il tempo mio giocondo;
	Tutto è follia, follia nel mondo
	ciò che non è piacer
	Godiam, fugace e rapido
	è il gaudio dell’amore,
	è un fior che nasce e muore,
	ne più si può goder
	Godiamo, c’invita, c’invita un fervido
	accento lusinghier.

	\songsection{Coro}
	Godiamo, la tazza, la tazza e il cantico,
	la notte abbella e il riso;
	in questo paradiso ne scopra il nuovo dì

	\songsection{Violetta:} La vita è nel tripudio

	\songsection{Alfredo:} Quando non s’ami ancora…

	\songsection{Violetta:} Nol dite a chi l’ignora,

	\songsection{Alfredo:} È il mio destin così…

	\songsection{Tutti}
	Godiamo, la tazza, la tazza e il cantico,
	la notte abbella e il riso;
	in questo paradiso ne scopra il nuovo dì.
\end{guitar}
