\songtitle{Ramund hin Unge}{Traditional}
\medskip

\begin{paracol}{2}
\small

\begin{guitar}
  \songsection{Verse 1}
  Ramund han var sig en bedre mand,
  om han havde bedre klæder.
  Dronningen gav hannem klæder på stand
  af blårgarn, bast og læder.
  ``Sådant vil jeg ikke ha','' sagde Ramund.
  ``Sådant står mig ikke bra,'' sagde RhU.

  \songsection{Verse 2}
  ``Bast og blårgarn er værre end ry,
  det kan du gi' tjenerne dine.''
  Frøkenen gav hannem klæderne ny
  af silke og sammet fine.
  ``Sådant vil jeg heller' ha','' sagde R.
  ``Sådant står mig meget bra,'' sagde RhU.

  \songsection{Verse 3}
  Ramund gik sig i stuen ind
  alt til den netteste skrædder.
  ``Hør, du skrædder, fager og fin,
  vil du gøre Ramund klæder?''
  ``Hvorfor sku' jeg ikke det?'' sagde skrædder'n
  ``Da gør du mege vel,'' sagde RhU.

  \songsection{Verse 4}
  Halvtredsindstyve alen tøj
  og femten til bukseremme.
  ``Det skal du gøre mig stærkt og drøjt,
  om du skal bukserne sømme.''
  ``Det er mig meget trangt,'' sagde R.
  ``Jeg kan ikke skræv' min gang,'' s. RhU.

  \songsection{Verse 5}
  Ramund takler skibene sine
  med udslagne sejl ved strande.
  Så sejled' han over salten sø
  alt inde under jætternes lande.
  ``Nu er vi komne her,'' sagde Ramund,
  ``foruden stort besvær,'' sagde RhU.

  \songsection{Verse 6}
  Ramund gik sig ved salten søstrand,
  der så han syv jætter stande.
  ``Tager jeg Ramund i min mindste hånd,
  jeg kaster ham langt fra lande.''
  ``Det gør aldrig ene du,'' sagde Ramund,
  ``komme må I alle syv,'' sagde RhU.
\end{guitar}

\switchcolumn

\begin{guitar}
  \songsection{Verse 7}
  Ramund tog til sit dyre sværd,
  det han kaldte Dymlingen røde.
  Hug han den jætter syv med én færd,
  at blodet randt dennem til døde.
  ``Der ligger alle syv,'' sagde Ramund.
  ``Alt står jeg her endnu,'' sagde RhU.

  \songsection{Verse 8}
  Ramund gik frem ved havets bred,
  der så han den store jætte.
  Halvtredsinstyve alen var han bred,
  og hundrede var vel hans længde.
  ``Vel er du bred og lang,'' sagde Ramund.
  ``Har du lyst at slås engang?'' sagde RhU.

  \songsection{Verse 9}
  ``Kære Ramund, du lade mig leve,
  og gøre mig ingen skade.
  Dig jeg giver syv tønder guld
  og vil dig i klaren vin bade.''
  ``Den ottende slunter med,'' sagde R.
  ``Dog sabler jeg dig med,'' sagde RhU.

  \songsection{Verse 10}
  Den første dag, de sammen drog,
  de kæmpedes med bare hænder.
  Ramund napped' i jættens skæg,
  at kødet det løsned's fra tænder.
  ``Så ilde griner du,'' sagde Ramund,
  ``og værre ser du ud,'' sagde RhU.

  \songsection{Verse 11}
  Den anden dag, de sammen lod stå,
  de ginge sammen med vrede.
  Det store stenbjerg, de stode på,
  de ned i ler monne træde.
  ``Den leg den er vel hård,'' sagde jætten.
  ``Vi begyndte først i går,'' sagde RhU.

  \songsection{Verse 12}
  Ramund tog til sit gode sværd,
  det han kaldte Dymlingen dyre.
  Hug han jættens hoved højt i vejr,
  som fire par øjne knap ku' røre.
  ``Jeg mente, den ej bed,'' sagde Ramund,
  ``den bed alligevel,'' sagde RhU.
\end{guitar}

\switchcolumn

\begin{guitar}
  \songsection{Verse 13}
  Ramund gik sig i bjerget ind
  til alle de små troldes sæde.
  Stride randt tårer de trolde på kind,
  de måtte for Ramund græde.
  ``Græder du for mig?'' sagde Ramund.
  ``Jeg græd aldrig for dig,'' sagde RhU.

  \songsection{Verse 14}
  Ramund han rykte og sloges omkring
  alt som de raskeste helte.
  Med alle de trolde han kørte i ring
  og dennem til jorden nedfældte.
  ``Herinde går det bravt,'' sagde Ramund.
  ``Det falder mig i lav,'' sagde RhU.

  \songsection{Verse 15}
  Ramund steg i skibet fuldt snart,
  det knaged' i hver en bunke.
  Alle de bådsmænd i skibet var,
  de tænkte, at de skulle sjunke.
  ``Vi sjunke ikke her,'' sagde Ramund.
  ``Vi sejle lige vel,'' sagde RhU.

  \songsection{Verse 16}
  Ramund ladede skibene syv
  med guld og med ædle stene.
  Sejled' så over sø så stiv
  alt ind under kejserens lene.
  ``Nu er vi komne her,'' sagde Ramund.
  ``Nu har vi bedre lært,'' sagde RhU.

  \songsection{Verse 17}
  Ramund kasted' anker på hviden sand
  og stavnen mod land lod svinge.
  Selv var Ramund den første mand,
  som ind på landet monne springe.
  ``Vover intet fler','' sagde Ramund.
  ``Vover intet mer','' sagde RhU.

  \songsection{Verse 18}
  Ramund gik ad boldhuset ind,
  der leged' de bold og guldterning.
  Alle forskrækked's for Ramunds skind
  og for hans grumme gebærder.
  ``En vakker leg er det,'' sagde Ramund.
  ``Får jeg vel lege med?'' sagde RhU.
\end{guitar}

\switchcolumn

\begin{guitar}
  \songsection{Verse 19}
  Kejseren ud af vinduet så
  med angest og sorrigfuld mine.
  ``Hvo er den mand, som i gården mon stå
  og dér så hæsselig grine?''
  ``Det er mig, og jeg har lyst,'' sagde R.
  ``med dig at vov' en dyst,'' sagde RhU.

  \songsection{Verse 20}
  Ramund slog på sit gode sværd,
  at jorden hun gungred' og rysted'.
  Fuglene dåned' og faldt ned på mark,
  som sjunge tilforn på kviste.
  ``Ret nu så bli'r jeg vred,'' sagde R.
  ``om ej du kommer ned,'' sagde RhU.

  \songsection{Verse 21}
  Ramund han gik til dørren ind,
  Med ivrigt sind og mod,
  Jeg skal nu ind i salen trine,
  Bar jern og stål end imod.
  ``agte dig nu Grant!'' sagde Ramund
  ``her bliver en hårder kamp'' sagde RhU.

  \songsection{Verse 22}
  Ramund han stødte på døren med stang,
  at hele slottet det revned'.
  Vindver og døre af væggen udsprang,
  og muren med jorden blev jævnet.
  ``Så du, at jeg slog ind?'' sagde Ramund.
  ``Det gælde vil dit skind!'' sagde RhU.

  \songsection{Verse 23}
  ``Kære Ramund, du lade mig leve,
  jeg så omend aldrig din lige.
  Min yngste datter vil jeg dig give,
  samt halvparten af mit rige.''
  ``Det ta'r jeg, når jeg vil,'' sagde R.
  ``og så din datter til,'' sagde RhU.

  \songsection{Verse 24}
  Ramund tog til sin store kniv,
  den han kaldte Dymlingens pile.
  Skilte han kejseren ved hans liv,
  at ho'det det fløj femten mile.
  ``Jeg troed', den ej bed,'' sagde Ramund.
  ``Dog rinder blodet ned,'' sagde RhU.
\end{guitar}
\end{paracol}
